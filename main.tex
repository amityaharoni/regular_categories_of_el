\documentclass[12pt]{article}
\usepackage{hyperref} % hyperlinks
\usepackage{graphicx} % images
\usepackage{amsmath} % math (e.g. \begin{align})
\usepackage{amssymb} % math symbols (e.g. \infty)
\usepackage{amsthm} % math (theorem environments)
\usepackage{amsfonts} % math (e.g. \mathbb)
\usepackage{amscd} % math for commutative diagrams (e.g. \xymatrix)
\usepackage{mathrsfs} % math for calligraphic letters (e.g. \mathscr)
\usepackage{mathtools} % math for \DeclarePairedDelimiter
\usepackage{stmaryrd} % math for \xhookrightarrow
\usepackage{tikz} % math for \tikz
\usepackage{./quiver} % quiver diagrams

\newtheorem{theorem}{Theorem}[section]
\newtheorem{lemma}[theorem]{Lemma}
\newtheorem{corollary}[theorem]{Corollary}
\newtheorem{proposition}[theorem]{Proposition}
\newtheorem{definition}[theorem]{Definition}
\newtheorem{example}[theorem]{Example}
\newtheorem{remark}[theorem]{Remark}
\newtheorem{claim}{Claim}

\newcommand{\EL}{\mathcal{EL}\text{++}}
\newcommand{\ELt}{$\mathcal{EL}$++}
\newcommand{\concepts}{{\bf C}}
\newcommand{\individuals}{{\bf A}}
\newcommand{\roles}{{\bf R}}
\newcommand{\tuple}[1]{\langle #1 \rangle}
\newcommand{\set}[1]{\{ #1 \}}
\newcommand{\abs}[1]{\lvert #1 \rvert}
\newcommand{\norm}[1]{\lVert #1 \rVert}
\newcommand{\ontology}{\mathcal{O}}
\newcommand{\ontologyt}{$\mathcal{O}$}
\newcommand{\interpetation}{\mathcal{I}}
\newcommand{\interpetationt}{$\mathcal{I}$}
\newcommand{\InfSemiLattice}{\text{InfSemiLattice}}
% Box space - written as Box
\newcommand{\bspace}{\mathsf{BoxSpace}}
\newcommand{\bobject}{\mathsf{Box}}

\begin{document}
\title{Summary of the article}
\author{Author}
\date{Date}
\maketitle

\section{Introduction}
\label{sec:intro}


\section{Preliminaries}
\label{sec:prelim}
\subsection{\ELt}
First we define the description logic \ELt using
set theory.
\subsubsection{Syntax}
\begin{definition}{Signature of \ELt}
    We define the signature of \ELt as a tuple
    \begin{align*}
        \sigma = \tuple{\concepts, \individuals, \roles}
    \end{align*}
    where $\concepts$ is a set of concept symbols, $\individuals$ is a set of
    individual symbols, and $\roles$ is a set of role symbols.
\end{definition}
\begin{definition}[Well formed concept]
    Given a signature $\sigma=\tuple{\concepts, \individuals, \roles}$, a
    well formed concept is a formula of the form
    \begin{align*}
        \varphi ::= \top | \bot | C | \{a\} | \varphi \sqcap \psi | 
        \exists r.\varphi | \varphi(a) | r(a,b)
    \end{align*}
    where $C \in \concepts$, 
    $a,b \in \individuals$, and $r \in \roles$.
    We denote the smallest 
    set of well formed concepts by $\mathcal{C}_\sigma$.
\end{definition}
\begin{definition}[Ontology]
    An ontology \ontologyt is a finite set
    of concept inclusions $C\sqsubseteq D$
    where $C,D \in \mathcal{C}_\sigma$. 
\end{definition}

\subsubsection{Semantics}
\begin{definition}[Interpretation]
    Given a signature $\sigma=\tuple{\concepts, \individuals, \roles}$,
    an interpretation \interpetationt is a tuple
    \begin{align*}
        \mathcal{I} = \tuple{\Delta^\mathcal{I}, \cdot^\mathcal{I}}
    \end{align*}
    where $\Delta^\mathcal{I}$ is a non-empty set called the domain of
    $\mathcal{I}$, and $\cdot^\mathcal{I}$ is a function that maps
    \begin{itemize}
        \item each $C \in \concepts$ to a subset $C^\mathcal{I}$ of
            $\Delta^\mathcal{I}$,
        \item each $a \in \individuals$ to an element $a^\mathcal{I}$ of
            $\Delta^\mathcal{I}$, and
        \item each $r \in \roles$ to a binary relation $r^\mathcal{I}$ on
            $\Delta^\mathcal{I}$.
    \end{itemize}
    such that for any $a,b\in\individuals$, $r\in\roles$, and $C,D\in\concepts$,
    we have that
    \begin{enumerate}
        \item $a^\mathcal{I} \in C^\mathcal{I}$ iff $\{a\} \sqsubseteq C$ iff $C(a)$,
        \item $a^\mathcal{I} \in \{a\}^\mathcal{I}$,
        \item $a^\mathcal{I} \in (C\sqcap D)^\mathcal{I}$ iff
            $a^\mathcal{I} \in C^\mathcal{I}$ and $a^\mathcal{I} \in D^\mathcal{I}$,
        \item $a^\mathcal{I} \in (\exists r.C)^\mathcal{I}$ iff
            there exists $b\in\individuals$ such that $(a^\mathcal{I}, b^\mathcal{I})\in r^\mathcal{I}$
            and $b^\mathcal{I} \in C^\mathcal{I}$,
        \item $(C\sqsubseteq D)^\mathcal{I}$ iff
            $C^\mathcal{I} \subseteq D^\mathcal{I}$,
        \item $a^\mathcal{I} \in r^\mathcal{I}(\{b\}^\mathcal{I})$ iff
            $r^\mathcal{I}(a^\mathcal{I},b^\mathcal{I})$,
        \item $a^\mathcal{I} \in \top^\mathcal{I}$,
        \item $a^\mathcal{I} \notin \bot^\mathcal{I}$.
    \end{enumerate}
\end{definition}
\begin{definition}[Satisfiability]
    Given a signature $\sigma=\tuple{\concepts, \individuals, \roles}$,
    an interpretation \interpetationt, and a concept $C\in\mathcal{C}_\sigma$,
    we say that $C$ is satisfiable in \interpetationt if there exists
    $a\in\individuals$ such that $a^\mathcal{I} \in C^\mathcal{I}$.
\end{definition}
\begin{definition}[Entailment]
    Given a signature $\sigma=\tuple{\concepts, \individuals, \roles}$,
    an interpretation \interpetationt, and an ontology \ontologyt,
    we say that \interpetationt satisfies \ontologyt, denoted by
    $\interpetation \models \ontology$, if for every $C\sqsubseteq D\in\ontology$,
    we have that $(C\sqsubseteq D)^\mathcal{I}$.
    We say that \ontologyt is satisfiable if there exists an interpretation
    \interpetationt such that $\interpetation \models \ontology$.
\end{definition}

\subsection{Regular categories}
To provide a framework for \ELt, we need to introduce regular logic.
First we define regular cateogries.

\begin{definition}[Regular Hyperdoctrine]
Let $\mathcal{C}$ be a finitely complete category.
A regular hyperdoctrine is
a functor
\begin{align*}
    P:\mathcal{C}^{\text{op}}\to \InfSemiLattice
\end{align*}
such that for any projection $\pi: A\times B\to C \in\mathcal{C}$,
the morphism $P(\pi)$, which we shall denote $\pi^{-1}$, has a left adjoint $\exists_\pi:P(C)\to P(A)\times P(B)$
%  and a right adjoint $\forall_\pi:P(C)\to P(A)\times P(B)$
 .
Meaning that for any two formulas $\varphi\in P(A)\times P(B),\psi\in P(C)$, we have that
\begin{gather}
    \exists_\pi x.\varphi\Rightarrow\psi\text{ if and only if }\varphi\Rightarrow\pi^{-1}(\psi)\label{eq:exists-adj}
    % \\
    % \pi^{-1}\varphi\Rightarrow\psi\text{ if and only if }\varphi\Rightarrow\forall_\pi\psi\label{eq:forall-adj}
\end{gather}
satisfying the Beck-Chevalley conditions and the Forbenius property.
\end{definition}
\begin{remark}
    Equation \ref*{eq:exists-adj} can be considered as a claim that $\exists_\pi x.\varphi\Rightarrow\psi$ if and only if there exists some $b\in B$
    such that $\varphi\Rightarrow\psi(b)$.
    % \\
    % Equation \ref*{eq:forall-adj} can be considered as a claim that  $\varphi\Rightarrow\forall_\pi\psi$ if and only if for all $b\in B$, $\varphi(b)\Rightarrow\psi$.
\end{remark}

\begin{definition}[Beck-Chevalley conditions]
    Given two projections $\pi:A\times B \to A$ and $\pi':C\times B \to C$,
    the Beck-Chevalley conditions state that for any two morphisms $f:A\times B\to C\times D$ and $g:A\to C$,
    such that we have a pullback
    \[\begin{tikzcd}
        {A\times B} && A \\
        \\
        {C\times B} && B
        \arrow["\pi"{description}, from=1-1, to=1-3]
        \arrow["g"{description}, from=1-3, to=3-3]
        \arrow["{\pi'}"{description}, from=3-1, to=3-3]
        \arrow["f"{description}, from=1-1, to=3-1]
    \end{tikzcd}\]
    then $\exists_\pi\circ P(f)=P(g)\circ\exists_{\pi'}$ 
    % and $\forall_\pi\circ P(f)=P(g)\circ\forall_{\pi'}$
    .
\end{definition}

\begin{definition}[Forbenius Condition]
    Given $\pi:A\times B\to B, \varphi\in P(A),\psi\in P(A)\times P(B)$, the Forbenius condition states that
    \begin{gather}
        \exists_\pi x.(\pi^{-1}(\varphi)\land\psi)\simeq \varphi\land\exists_\pi x.\psi
    \end{gather}
    This can be read as: there exists $a\in A$
    such that $ \varphi(a)\land\exists_\pi x.\psi(a,x)$,
if and only if $\exists_\pi x.(\varphi(a)\land\psi(a,x))$
\end{definition}

The logic of regular hyperdoctrines is regular logic. It is the $\exists-\land$ fragment
of first order logic.
% \begin{remark}
%     As the image of $P$ is a lattice, $P(f)$ is monotone and so captures the right notion of substitution:
%     Given morphism $f:A\to B$, objects $\varphi,\psi\in PB$, we have that
%     \[
%         \text{if }\varphi\Rightarrow\psi\text{ then }P(f)(\varphi)\Rightarrow P(f)(\psi)
%     \]
% \end{remark}
% \begin{example}[The category of sets]
%     The category of sets is a regular hyperdoctrine.
%     The functor $P$ is the powerset functor,
%     taking a set $A$ to the set of all subsets (equations) of $A$
%     with the ordering given by inclusion and meets given by intersection.
%     \par Given a projection $\pi:A\times B\to A$,
%     we have that
%     $\pi^{-1}:PA\to PA\times PB$ is a map that takes a subset $S\subseteq A$
%     to the subset of all pairs $(a,b)\in A\times B$ such that $a\in S$.
%     If we consider $S$ as a formula $\varphi(a,x)$ then $\pi^{-1}(\varphi(a,x))$ is 
%     the formula $\varphi(a,x)[b/x]$.
%     \par Given a morphism $f:A\to B$,
%     the morphism $P(f)$ represents that
%     if $\varphi(b)$ and if $a$ is such that $f(a)=b$, then $\varphi(a)$.
%     Thus we have that
%     \[
%         \text{if }\varphi\Rightarrow\psi\text{ then }\varphi(f(a))\Rightarrow\psi(f(a))
%     \]
%     \par Given a projection $\pi:(a,b)\to a$,
%     we have that 
%     \[
%         \pi^{-1}:\varphi\to\{(a,b):\varphi(a)\}
%     \]
%     is the inverse image of $\varphi$ under $\pi$.
%     \par The left adjoint of $\pi^{-1}$ is given by
%     \[
%         \exists_\pi:\psi\to\{a:\exists b.\psi(a,b)\}
%     \]
%     which is the set of all $a$ such that there exists some $b$ such that $\psi(a,b)$.
    
%     \par Given a pullback square
%     \[\begin{tikzcd}
%         {A\times B} && B \\
%         \\
%         {A\times C} && C
%         \arrow["\pi"{description}, from=1-1, to=1-3]
%         \arrow["g"{description}, from=1-3, to=3-3]
%         \arrow["{\pi'}"{description}, from=3-1, to=3-3]
%         \arrow["f"{description}, from=1-1, to=3-1]
%     \end{tikzcd}\]
%     The Beck-Chevalley condition states that for any $\varphi\in P(A\times C)$,
    

% \end{example}

\section{Main results}
\label{sec:main}

\begin{definition}[Box space]
Given $\mathbb{R}^n$, we define a box space $\mathcal{B}$ as a set
of subsets of the form
$\{x_i:m_i\leq x_i\leq M_i\}_{\vec{m}}^{\vec{M}}$ that is closed under arbitrary intersections and contains $\mathbb{R}^n$ and $\varnothing$. We would
denote members as the set as $\bobject(\vec{m},\vec{M}):=\{x_i:m_i\leq x_i\leq M_i\}_{\vec{m}}^{\vec{M}},\bobject(-\infty,\infty),\varnothing$.
\end{definition}
We can observe that the subset relation induces a well-ordering of
$(\mathbb{R}^n,\mathcal{B})$ and that space is bounded, thus box spaces are
infimum lattices.
\begin{definition}[Box space morphism]
    Given two box spaces $(\mathbb{R}^n,\mathcal{B})$ and $(\mathbb{R}^m,\mathcal{B}')$,
    we define a box space morphism $f:(\mathbb{R}^n,\mathcal{B})\to(\mathbb{R}^m,\mathcal{B}')$ as a function
    $f:\mathbb{R}^n\to\mathbb{R}^m$ such that for all $\bobject(\vec{m},\vec{M})\in(\mathbb{R}^n,\mathcal{B})$,
    $f(\bobject(\vec{m},\vec{M}))\in(\mathbb{R}^m,\mathcal{B}')$ and if $\bobject(\vec{m},\vec{M})\subseteq\bobject(\vec{m'},\vec{M'})$
    then $f(\bobject(\vec{m},\vec{M}))\subseteq f(\bobject(\vec{m'},\vec{M'}))$.
\end{definition}
This induces a category $\mathbf{Box}$ of box spaces and box space morphisms.
\begin{definition}[Product box space]
    Given two box spaces $(\mathbb{R}^n,\mathcal{B})$ and $(\mathbb{R}^m,\mathcal{B}')$,
    we define their product box space $\mathcal{B}\times\mathcal{B}'$ as the set of all subsets of the form
    $\bobject(m\times m',M\times M')=(\bobject(\vec{m},\vec{M}),\bobject(\vec{m'},\vec{M'})):=\{(\vec{x}, \vec{x'}):m_i\leq x_i\leq M_i\land m'_i\leq x'\leq M'_i\}$. 
\end{definition}
To see that this is a box space,
    notice that $\bigcap_j (\bobject(\vec{m},\vec{M}),\bobject(\vec{m'},\vec{M'}))_j=\bigcap_j \bobject(\vec{m},\vec{M})_j \times \bigcap_j \bobject(\vec{m'},\vec{M'})_j$.
\begin{lemma}
    The product box space is the product in $\mathbf{Box}$.
\end{lemma}
\begin{claim}
    The map $\exists_\pi:\mathcal{B}\times\mathcal{B}'\to\mathcal{B}$ given by
    \[
        \exists_\pi(\bobject(\vec{m},\vec{M}),\bobject(\vec{m'},\vec{M'}))=\{\vec{x}:\exists \vec{x'}\in \bobject(\vec{m'},\vec{M'}):(\vec{x},\vec{x'})
        \in \bobject(\vec{m}\times \vec{m'},\vec{M}\times \vec{M'})
        \}
    \]
    is a box space morphism and the left adjoint of $\pi^{-1}$.
\end{claim}
\end{document}

% predicates and relations are only binary relations
% existential is exists b.(a,b)\in R and b\in C
% Add examples of types and of sets
% Beck-Chevalley condition is essentially that substitution commutes with the and operator

% boxes would not be graphs of affine transformations but rather boxes in higher dimension
% This is in the sense that they are predicates of higher dimensional