\documentclass[12pt]{article}
\usepackage{hyperref} % hyperlinks
\usepackage{graphicx} % images
\usepackage{amsmath} % math (e.g. \begin{align})
\usepackage{amssymb} % math symbols (e.g. \infty)
\usepackage{amsthm} % math (theorem environments)
\usepackage{amsfonts} % math (e.g. \mathbb)
\usepackage{amscd} % math for commutative diagrams (e.g. \xymatrix)
\usepackage{mathrsfs} % math for calligraphic letters (e.g. \mathscr)
\usepackage{mathtools} % math for \DeclarePairedDelimiter
\usepackage{stmaryrd} % math for \xhookrightarrow
\usepackage{tikz} % math for \tikz
\usepackage{./quiver} % quiver diagrams


\newtheorem{theorem}{Theorem}[section]
\newtheorem{lemma}[theorem]{Lemma}
\newtheorem{corollary}[theorem]{Corollary}
\newtheorem{proposition}[theorem]{Proposition}
\newtheorem{definition}[theorem]{Definition}
\newtheorem{example}[theorem]{Example}
\newtheorem{remark}[theorem]{Remark}
\newtheorem{claim}{Claim}

\newcommand{\EL}{\mathcal{EL}\text{++}}
\newcommand{\ELt}{$\mathcal{EL}$++}
\newcommand{\concepts}{{\bf C}}
\newcommand{\individuals}{{\bf A}}
\newcommand{\roles}{{\bf R}}
\newcommand{\tuple}[1]{\langle #1 \rangle}
\newcommand{\set}[1]{\{ #1 \}}
\newcommand{\abs}[1]{\lvert #1 \rvert}
\newcommand{\norm}[1]{\lVert #1 \rVert}
\newcommand{\ontology}{\mathcal{O}}
\newcommand{\ontologyt}{$\mathcal{O}$}
\newcommand{\interpetation}{\mathcal{I}}
\newcommand{\interpetationt}{$\mathcal{I}$}
\newcommand{\InfSemiLattice}{\text{InfSemiLattice}}
% Box space - written as Box
\newcommand{\bspace}{\mathsf{BoxSpace}}
\newcommand{\bobject}{\mathsf{Box}}

\begin{document}
\title{Summary of the article}
\author{Author}
\date{Date}
\maketitle

\section{Introduction}
\label{sec:intro}


\section{Preliminaries}
\label{sec:prelim}
\subsection{\ELt}
First we define the description logic \ELt using
set theory.
\subsubsection{Syntax}
\begin{definition}{Signature of \ELt}
    We define the signature of \ELt as a tuple
    \begin{align*}
        \sigma = \tuple{\concepts, \individuals, \roles}
    \end{align*}
    where $\concepts$ is a set of concept symbols, $\individuals$ is a set of
    individual symbols, and $\roles$ is a set of role symbols.
\end{definition}
\begin{definition}[Well formed concept]
    Given a signature $\sigma=\tuple{\concepts, \individuals, \roles}$, a
    well formed concept is a formula of the form
    \begin{align*}
        \varphi ::= \top | \bot | C | \{a\} | \varphi \sqcap \psi | 
        \exists r.\varphi | \varphi(a) | r(a,b)
    \end{align*}
    where $C \in \concepts$, 
    $a,b \in \individuals$, and $r \in \roles$.
    We denote the smallest 
    set of well formed concepts by $\mathcal{C}_\sigma$.
\end{definition}
\begin{definition}[Ontology]
    An ontology \ontologyt is a finite set
    of concept inclusions $C\sqsubseteq D$
    where $C,D \in \mathcal{C}_\sigma$. 
\end{definition}


\subsubsection{Semantics}
\begin{definition}[Interpretation]
    Given a signature $\sigma=\tuple{\concepts, \individuals, \roles}$,
    an interpretation \interpetationt is a tuple
    \begin{align*}
        \mathcal{I} = \tuple{\Delta^\mathcal{I}, \cdot^\mathcal{I}}
    \end{align*}
    where $\Delta^\mathcal{I}$ is a non-empty set called the domain of
    $\mathcal{I}$, and $\cdot^\mathcal{I}$ is a function that maps
    \begin{itemize}
        \item each $C \in \concepts$ to a subset $C^\mathcal{I}$ of
            $\Delta^\mathcal{I}$,
        \item each $a \in \individuals$ to an element $a^\mathcal{I}$ of
            $\Delta^\mathcal{I}$, and
        \item each $r \in \roles$ to a binary relation $r^\mathcal{I}$ on
            $\Delta^\mathcal{I}$.
    \end{itemize}
    such that for any $a,b\in\individuals$, $r\in\roles$, and $C,D\in\concepts$,
    we have that
    \begin{enumerate}
        \item $a^\mathcal{I} \in C^\mathcal{I}$ iff $\{a\} \sqsubseteq C$ iff $C(a)$,
        \item $a^\mathcal{I} \in \{a\}^\mathcal{I}$,
        \item $a^\mathcal{I} \in (C\sqcap D)^\mathcal{I}$ iff
            $a^\mathcal{I} \in C^\mathcal{I}$ and $a^\mathcal{I} \in D^\mathcal{I}$,
        \item $a^\mathcal{I} \in (\exists r.C)^\mathcal{I}$ iff
            there exists $b\in\individuals$ such that $(a^\mathcal{I}, b^\mathcal{I})\in r^\mathcal{I}$
            and $b^\mathcal{I} \in C^\mathcal{I}$,
        \item $(C\sqsubseteq D)^\mathcal{I}$ iff
            $C^\mathcal{I} \subseteq D^\mathcal{I}$,
        \item $a^\mathcal{I} \in r^\mathcal{I}(\{b\}^\mathcal{I})$ iff
            $r^\mathcal{I}(a^\mathcal{I},b^\mathcal{I})$,
        \item $a^\mathcal{I} \in \top^\mathcal{I}$,
        \item $a^\mathcal{I} \notin \bot^\mathcal{I}$.
    \end{enumerate}
\end{definition}
\begin{definition}[Satisfiability]
    Given a signature $\sigma=\tuple{\concepts, \individuals, \roles}$,
    an interpretation \interpetationt, and a concept $C\in\mathcal{C}_\sigma$,
    we say that $C$ is satisfiable in \interpetationt if there exists
    $a\in\individuals$ such that $a^\mathcal{I} \in C^\mathcal{I}$.
\end{definition}
\begin{definition}[Entailment]
    Given a signature $\sigma=\tuple{\concepts, \individuals, \roles}$,
    an interpretation \interpetationt, and an ontology \ontologyt,
    we say that \interpetationt satisfies \ontologyt, denoted by
    $\interpetation \models \ontology$, if for every $C\sqsubseteq D\in\ontology$,
    we have that $(C\sqsubseteq D)^\mathcal{I}$.
    We say that \ontologyt is satisfiable if there exists an interpretation
    \interpetationt such that $\interpetation \models \ontology$.
\end{definition}

\subsection{Regular categories}
To provide a framework for \ELt, we need to introduce regular logic.
First we define regular cateogries.

\begin{definition}[Regular Hyperdoctrine]
Let $\mathcal{C}$ be a finitely complete category.
A regular hyperdoctrine is
a functor
\begin{align*}
    P:\mathcal{C}^{\text{op}}\to \InfSemiLattice
\end{align*}
such that for any morphism $\pi: A\to C \in\mathcal{C}$,
the morphism $P(\pi)$, which we shall denote $\pi^{-1}$, has a left adjoint $\exists_\pi:P(A)\to P(C)$
%  and a right adjoint $\forall_\pi:P(C)\to P(A)\times P(B)$
 .
Meaning that for any two formulas $\varphi\in P(A),\psi\in P(C)$, we have that
\begin{gather}
    \exists_\pi x.\varphi\Rightarrow\psi\text{ if and only if }\varphi\Rightarrow\pi^{-1}(\psi)\label{eq:exists-adj}
    % \\
    % \pi^{-1}\varphi\Rightarrow\psi\text{ if and only if }\varphi\Rightarrow\forall_\pi\psi\label{eq:forall-adj}
\end{gather}
satisfying the Beck-Chevalley conditions and the Forbenius property.
\end{definition}
\begin{remark}
When we consider products and projections $\pi:A\times B\to C$, the adjoint condition becomes the 
natural existential quantifier.
Equation \ref*{eq:exists-adj} can then be considered as a claim that 
$\exists_\pi x.\varphi\Rightarrow\psi$ if and only if there exists some $b\in B$
such that $\varphi\Rightarrow\psi(b)$.
    % \\
    % Equation \ref*{eq:forall-adj} can be considered as a claim that  $\varphi\Rightarrow\forall_\pi\psi$ if and only if for all $b\in B$, $\varphi(b)\Rightarrow\psi$.
\end{remark}

\begin{definition}[Beck-Chevalley conditions]
    Given two projections $\pi:A\times B \to A$ and $\pi':C\times B \to C$,
    the Beck-Chevalley conditions state that for any two morphisms $f:A\times B\to C\times D$ and $g:A\to C$,
    such that we have a pullback
    \[\begin{tikzcd}
        {A\times B} && A \\
        \\
        {C\times B} && B
        \arrow["\pi"{description}, from=1-1, to=1-3]
        \arrow["g"{description}, from=1-3, to=3-3]
        \arrow["{\pi'}"{description}, from=3-1, to=3-3]
        \arrow["f"{description}, from=1-1, to=3-1]
    \end{tikzcd}\]
    then $\exists_\pi\circ P(f)=P(g)\circ\exists_{\pi'}$ 
    % and $\forall_\pi\circ P(f)=P(g)\circ\forall_{\pi'}$
    .
\end{definition}

\begin{definition}[Forbenius Condition]
    Given $\pi:A\times B\to B, \varphi\in P(A),\psi\in P(A)\times P(B)$, the Forbenius condition states that
    \begin{gather}
        \exists_\pi x.(\pi^{-1}(\varphi)\land\psi)\simeq \varphi\land\exists_\pi x.\psi
    \end{gather}
    This can be read as: there exists $a\in A$
    such that $ \varphi(a)\land\exists_\pi x.\psi(a,x)$,
if and only if $\exists_\pi x.(\varphi(a)\land\psi(a,x))$
\end{definition}

The logic of regular hyperdoctrines is regular logic. It is the $\exists-\land$ fragment
of first order logic.
% \subsection{Moore Closure}
% Moore closure
% make some results nicer.
% \begin{definition}[Moore closure]
%     Let $X$ be a set and $\mathcal{L}$ be a set of subsets of $X$.
%     The Moore closure of $\mathcal{L}$, denoted $\mathcal{L}^*$, 
%     is the smallest set of subsets of $X$ such that
%     \begin{enumerate}
%         \item $\mathcal{L}\subseteq\mathcal{L}^*$.
%         \item $\mathcal{L}^*$ is closed under arbitrary intersections.
%     \end{enumerate}
% \end{definition}
% \begin{remark}
%     As the image of $P$ is a lattice, $P(f)$ is monotone and so captures the right notion of substitution:
%     Given morphism $f:A\to B$, objects $\varphi,\psi\in PB$, we have that
%     \[
%         \text{if }\varphi\Rightarrow\psi\text{ then }P(f)(\varphi)\Rightarrow P(f)(\psi)
%     \]
% \end{remark}
% \begin{example}[The category of sets]
%     The category of sets is a regular hyperdoctrine.
%     The functor $P$ is the powerset functor,
%     taking a set $A$ to the set of all subsets (equations) of $A$
%     with the ordering given by inclusion and meets given by intersection.
%     \par Given a projection $\pi:A\times B\to A$,
%     we have that
%     $\pi^{-1}:PA\to PA\times PB$ is a map that takes a subset $S\subseteq A$
%     to the subset of all pairs $(a,b)\in A\times B$ such that $a\in S$.
%     If we consider $S$ as a formula $\varphi(a,x)$ then $\pi^{-1}(\varphi(a,x))$ is 
%     the formula $\varphi(a,x)[b/x]$.
%     \par Given a morphism $f:A\to B$,
%     the morphism $P(f)$ represents that
%     if $\varphi(b)$ and if $a$ is such that $f(a)=b$, then $\varphi(a)$.
%     Thus we have that
%     \[
%         \text{if }\varphi\Rightarrow\psi\text{ then }\varphi(f(a))\Rightarrow\psi(f(a))
%     \]
%     \par Given a projection $\pi:(a,b)\to a$,
%     we have that 
%     \[
%         \pi^{-1}:\varphi\to\{(a,b):\varphi(a)\}
%     \]
%     is the inverse image of $\varphi$ under $\pi$.
%     \par The left adjoint of $\pi^{-1}$ is given by
%     \[
%         \exists_\pi:\psi\to\{a:\exists b.\psi(a,b)\}
%     \]
%     which is the set of all $a$ such that there exists some $b$ such that $\psi(a,b)$.
    
%     \par Given a pullback square
%     \[\begin{tikzcd}
%         {A\times B} && B \\
%         \\
%         {A\times C} && C
%         \arrow["\pi"{description}, from=1-1, to=1-3]
%         \arrow["g"{description}, from=1-3, to=3-3]
%         \arrow["{\pi'}"{description}, from=3-1, to=3-3]
%         \arrow["f"{description}, from=1-1, to=3-1]
%     \end{tikzcd}\]
%     The Beck-Chevalley condition states that for any $\varphi\in P(A\times C)$,
    

% \end{example}

\section{Main results}
\label{sec:main}
\subsection{Box spaces}
We first construct the box space category.
A box is a subset of $\mathbb{R}^n$ of the form 
$\{\vec{x}:m_i\leq x_i\leq M_i\}$ where for any $i$,
$m_i,M_i\in\mathbb{R}\cup\{-\infty,\infty\}$ and $m_i\leq M_i$.
Denote sets of that form $\bobject(\vec{m},\vec{M})$.
Note that $\bobject(-\infty,\infty)=S$
 and $\bobject(-\infty,-\infty)=\bobject(\infty,\infty)=\varnothing$.
\begin{definition}[Box space]
Given $S\subseteq\mathbb{R}^n$, we define a box space $\mathcal{B}$ as a set
of subsets of $S$ of the form
$\bobject(\vec{m},\vec{M})$ where
\begin{itemize}
    \item $S,\varnothing\in\mathcal{B}$.
    \item $\mathcal{B}$ is closed under arbitrary intersections.
\end{itemize}
\end{definition}
% \begin{remark}[Moore closure and the $\bspace$ monads]
%     We can observe that the subset relation induces a well-ordering of
% $(\mathbb{R}^n,\mathcal{B})$ and that space is bounded. In fact the box spaces are
% a closure system on $\mathbb{R}^n$. Meaning that they $\mathcal{B}$
% are complete lattices.
% \par In terms of Moore closure, $Cl(A)=\bigcap\{\bobject(m,M):A\subseteq \bobject(m,M)\}$ can be seen as a closure operator 
% since it is isotone, extensive, and idempotent.
% Thus box spaces are monads on the subobject lattice $\mathcal{P}(\mathbb{R}^n)$
% with maps
% \begin{itemize}
%     \item $\eta:A\to Cl(A)$
%     \item $\mu:Cl(Cl(A))\to Cl(A)$
% \end{itemize}
% We also observe that the box spaces are not topological spaces since closure systems are 
% topological spaces only if $Cl(A\cup B)=Cl(A)\cup Cl(B)$.
% \par Furthermore, we have a map from box spaces to lattices:
% \[
% (\mathbb{R}^n,\mathcal{B})\to(\mathcal{B},\cap,\subseteq)   
% \]
% \end{remark}

% \begin{remark}[Moore closure]
%     \par I would like us to cover Moore closures in the meeting if possible.
%     I hoped that this would generate the morphisms in the category of box space naturally
%     and thus provide a motivation for defining the box space morphisms.
%     \par I also hoped that this would provide a notion of proudct.
% \end{remark}
Notice that $\mathcal{B}$ is a meet-semilattice under intersections with the subset relation. 
\begin{definition}[Box space morphism]
    Given two box spaces $(S,\mathcal{B})$ and $(S',\mathcal{B}')$,
    we define a box space morphism $f:(S,\mathcal{B})\to(S',\mathcal{B}')$ as a function
    $f:S\to S'$ such that for all $\bobject(\vec{m},\vec{M})\in\mathcal{B}'$,
    $f^{-1}[\bobject(\vec{m},\vec{M})]\in\mathcal{B}$.
\end{definition}
Due to the distributivity of intersection over set inverses, we have that
\begin{lemma}
    Every box space morphism 
    $f:(S,\mathcal{B})\to(S',\mathcal{B}')$ induces a
    meet-semilattice morphism $f^{-1}:\mathcal{B}'\to\mathcal{B}$.
\end{lemma}
This induces a category $\mathbf{Box}$ of box spaces and box space morphisms.
Furthermore, this induces a forgetful functor $U:\mathbf{Box}^{op}\to\InfSemiLattice$:
\begin{gather*}
    U_0:(S,\mathcal{B})\mapsto\mathcal{B}
    \\
    U_1:f(\cdot)\mapsto f^{-1}[\cdot]
\end{gather*}
We shall show that this induces a hyperdoctrine.
\subsubsection{Finite completeness} 
We begin by showing that
$\mathbf{Box}$ is a finitely complete category.
We shall show this 
by proving that $\mathbf{Box}$ admits products and equalizers.
\par Define the product of two boxes:
    \begin{gather*}
        \bobject(\vec{m},\vec{M})\times\bobject(\vec{m'},\vec{M'})=
        \bobject(\vec{m}\times\vec{m'},\vec{M}\times\vec{M'})
    \end{gather*}
\begin{definition}[Product box space]
    Given two box spaces $(S,\mathcal{B})$ and $(S',\mathcal{B}')$,
    we define their product box space 
    $(S\times S',\mathcal{B}\times\mathcal{B}')$ 
    as the set of all subsets of 
    the form
    $\bobject(\vec{m},\vec{M})\times\bobject(\vec{m'},\vec{M'})$
    with $\bobject(\vec{m},\vec{M})\in\mathcal{B}$ 
    and 
    $\bobject(\vec{m'},\vec{M'})\in \mathcal{B}'$.
\end{definition}
\begin{lemma}
    The product box space is the product in $\mathbf{Box}$.
    \begin{proof}
        As every
        box morphism is also a function of the underlying sets,
        it suffcies to show that the
        projections $\pi_{S},\pi'_{S'}$
        and the unique morphisms
        are box morphisms.
        \par Given two box spaces $(S,\mathcal{B})$ and $(S',\mathcal{B}')$,
        and $\bobject(\vec{m},\vec{M})\in(\mathbb{R}^n,\mathcal{B})$, notice that
        $\pi^{-1}[\bobject(\vec{m},\vec{M})]=\bobject(\vec{m},\vec{M})\times\bobject(-\infty,\infty)\in\mathcal{B}\times\mathcal{B}'$.
        Thus the projections are box morphisms.
        \par Given a box space $(A,\mathcal{S})$
        and two morphisms $f:(A,\mathcal{S})\to(S,\mathcal{B})$
        and $g:(A,\mathcal{S})\to(S',\mathcal{B}')$,
        the unique map of sets $f\times g:x\to(f(x),g(x))$ that commutes with the projections is also a box morphism.
        This is because 
        \begin{gather*}
            f\times g^{-1}[\bobject(\vec{m},\vec{M})\times\bobject(\vec{m'},\vec{M'})]=
            f^{-1}[\bobject(\vec{m},\vec{M})]\cap g^{-1}[\bobject(\vec{m'},\vec{M'})]
        \end{gather*}
        As $f$ and $g$ are box morphisms, the preimages are in the box space $(\mathbb{R}^s,\mathcal{S})$.
        As box spaces are closed under intersection, $f^{-1}[\bobject(\vec{m},\vec{M})]\cap g^{-1}[\bobject(\vec{m'},\vec{M'})]$
        is in the box space. Thus, $f\times g$ is a box morphism.
    \end{proof}
\end{lemma}
Next we define the equalizer of two box space morphisms. First, we define a box subspace
and then we shall show that it is the equalizer.
\begin{definition}[Box subspace]
    Given a box space $(S,\mathcal{B})$ and a subset $A\subseteq S$,
    we define the box subsapce $(A,\mathcal{B}_A)$ as the set of all subsets of $A$
    of the form $\bobject(\vec{m},\vec{M})\cap A$ where $\bobject(\vec{m},\vec{M})\in\mathcal{B}$.
\end{definition}
Notice that this is a valid box space since it is closed under arbitrary intersections by the 
closure of $\mathcal{B}$. Furthermore, the inclusion $i:(A,\mathcal{B}_A)\to(S,\mathcal{B})$
is a box morphism.
\begin{definition}[Equalizer box space]
    Given two box spaces $(S,\mathcal{B})$ and $(S',\mathcal{B}')$,
    and two box space morphisms $f,g:(S,\mathcal{B})\to(S',\mathcal{B}')$,
    we define their equalizer box space
    $i:(E,\mathcal{B}_E)\to(S,\mathcal{B})$ as the set $\{\vec{x}:f(x)=g(x)\}$ with the inclusion
    map and the box subspace.
\end{definition}
\begin{lemma}
    The equalizer box space is the equalizer in $\mathbf{Box}$
    \begin{proof}
        Let $(A,\mathcal{S})$ be a box space with a morphism $j:(A,\mathcal{S})\to(S,\mathcal{B})$
        such that $fj=gj$. As $E$ is the equalizer in $\mathbf{Set}$, 
        it suffices to show that the unique map
        $j\restriction_E$ is a box morphism. Given a box $\bobject(\vec{m},\vec{M})$ in $\mathcal{B}_E$, we know that
        $\bobject(\vec{m},\vec{M})=\bobject(\vec{m'},\vec{M'})\cap S$ with $\bobject(\vec{m'},\vec{M'})\in\mathcal{S}$.
        As preimages commute with the intersection
        \begin{gather*}
            j\restriction_E^{-1}[\bobject(\vec{m},\vec{M})]=j^{-1}[\bobject(\vec{m'},\vec{M'})\cap S]=
            j^{-1}[\bobject(\vec{m'},\vec{M'})]\cap j^{-1}[S]=j^{-1}[\bobject(\vec{m'},\vec{M'})]
        \end{gather*}
        with the right hand side being a box in $\mathcal{S}$ because $j$ is a box morphism.
        Thus $j\restriction_E$ is a box morphism.
    \end{proof}
\end{lemma}
Lastly, note that $(0,\{0,\varnothing\})$ is a terminal object in $\mathbf{Box}$. Since we
showed that the category has binary products, equalizers and a terminal object, we showed that
\begin{theorem}
    $\mathbf{Box}$ is a finitely complete category.
\end{theorem}
\subsubsection{Left adjoint} 
% \begin{claim}
%     The map $\exists_\pi:\mathcal{B}\times\mathcal{B}'\to\mathcal{B}$ given by
%     \[
%         \exists_\pi(\bobject(\vec{m},\vec{M}),\bobject(\vec{m'},\vec{M'}))=\{\vec{x}:\exists \vec{x'}\in \bobject(\vec{m'},\vec{M'}):(\vec{x},\vec{x'})
%         \in \bobject(\vec{m}\times \vec{m'},\vec{M}\times \vec{M'})
%         \}
%     \]
%     is a box space morphism and the left adjoint of $\pi^{-1}$.
% \end{claim}
To define the existential object, it suffices to show that $\mathbf{Box}$
has a closure operator.
\begin{definition}[Closure operator]
    Given a box space $(S,\mathcal{B})$, we define
    \begin{align*}
        Cl:\mathcal{P}(S)&\to\mathcal{B}
        \\
        A&\mapsto\bigcap\{\bobject(\vec{m},\vec{M}):A\subseteq\bobject(\vec{m},\vec{M})\}
    \end{align*}
\end{definition}
We notice that the closure operator satisfies the closure conditions: for any $A\subseteq S$, we have that
\begin{enumerate}
    \item[(Extensivity)] $A\subseteq Cl(A)$.
    \item[(Isotonicity)] $A\subseteq B$ implies $Cl(A)\subseteq Cl(B)$.
    \item[(Idempotency)] $Cl(Cl(A))\subseteq Cl(A)$.
\end{enumerate}
Furthermore, for any box $\bobject(\vec{m},\vec{M})\in\mathcal{B}$, we have that
$Cl(\bobject(\vec{m},\vec{M}))=\bobject(\vec{m},\vec{M})$.
\begin{definition}[Existential object]
    We define the existential operator as:
    \begin{align*}
        \exists_{(\cdot)}:\mathbf{Box}((S,\mathcal{B}),(S',\mathcal{B}'))&\to\InfSemiLattice(\mathcal{B},\mathcal{B}')
        \\
        f(\cdot)&\mapsto Cl\circ f[\cdot]
    \end{align*}
    To see that this is a box morphism, notice that for any $\bobject(\vec{m},\vec{M})\in\mathcal{B}'$,
    \begin{gather*}
        (Cl\circ f)^{-1}[\bobject(\vec{m},\vec{M})]=f^{-1}\circ Cl^{-1}(\bobject(\vec{m},\vec{M}))=
        f^{-1}\left[\bigcup \{A\subseteq S':Cl(A)=\bobject(\vec{m},\vec{M})\}\right]
    \end{gather*}
    Notice that $Cl(\bobject(\vec{m},\vec{M}))=\bobject(\vec{m},\vec{M})$ and in fact
    it is the supremum of the set in the left. Thus
    \begin{gather*}
        =f^{-1}[\bobject(\vec{m},\vec{M})]
    \end{gather*}
    which is a box by the fact the $f$ is a box morphism.
\end{definition}
\begin{lemma}
    For any morphism $\pi:(S,\mathcal{B})\to(S',\mathcal{B}')$,
    the existential operator $\exists_\pi$ is the left adjoint of $\pi^{-1}$ in $\InfSemiLattice$.
    \begin{proof}
        We shall show that for any $\bobject(\vec{m},\vec{M})\in\mathcal{B}$ and $\bobject(\vec{m'},\vec{M'})\in\mathcal{B}'$,
        \begin{gather*}
            \exists_\pi[\bobject(\vec{m},\vec{M})]\subseteq \bobject(\vec{m'},\vec{M'})\text{ if and only if }
            \bobject(\vec{m},\vec{M})\subseteq \pi^{-1}[\bobject(\vec{m'},\vec{M'})]
        \end{gather*}
        \par $\Rightarrow$ Suppose that $Cl\circ \pi[\bobject(\vec{m},\vec{M})]=\exists_\pi[\bobject(\vec{m},\vec{M})]\subseteq \bobject(\vec{m'},\vec{M'})$.
        Let $\vec{x}\in\bobject(\vec{m},\vec{M})$.
        Then by extensivity $\pi(\vec{x})\in\pi[\bobject(\vec{m},\vec{M})]\subseteq Cl\circ \pi[\bobject(\vec{m},\vec{M})]$.
        Hence $\pi(\vec{x})\in\bobject(\vec{m'},\vec{M'})$. Thus $\vec{x}\in\pi^{-1}[\bobject(\vec{m'},\vec{M'})]$.
        \par $\Leftarrow$ Suppose that $\bobject(\vec{m},\vec{M})\subseteq \pi^{-1}[\bobject(\vec{m'},\vec{M'})]$.
        Then $\pi[\bobject(\vec{m},\vec{M})]\subseteq\bobject(\vec{m'},\vec{M'})$.
        By isotonicity $Cl(\pi[\bobject(\vec{m},\vec{M})])\subseteq Cl(\bobject(\vec{m'},\vec{M'}))=\bobject(\vec{m'},\vec{M'})$.
    \end{proof}
\end{lemma}
\begin{remark}
    Given a product box space and a projection $\pi:(S\times S',\mathcal{B}\times\mathcal{B}')\to(S,\mathcal{B})$,
    we have in particular that
    \begin{gather*}
        \exists_\pi(\bobject(\vec{m},\vec{M})\times\bobject(\vec{m'},\vec{M'}))=
        \{\vec{x}:\exists \vec{x'}\in \bobject(\vec{m'},\vec{M'}):(\vec{x},\vec{x'})
        \in \bobject(\vec{m},\vec{M})\times\bobject(\vec{m'},\vec{M'})
        \}
    \end{gather*}
\end{remark}
We need to show
that the existential operator satisfies the Beck-Chevalley
and the Forbenius conditions.
First, note that
\begin{lemma}[Characterization of the pullback]\label{lem:box-pullback}
    Given any two arrows $f:(A,\mathcal{A})\to(C,\mathcal{C})$
    and $g:(B,\mathcal{B})\to(C,\mathcal{C})$ in the category $\mathbf{Box}$,
    the pullback is of the form
    $(A\times_{f,g} B,\mathcal{A\times B}_{A\times_{f,g} B})$. Meaning that 
    the underlying set is the pullback set in $\mathbf{Set}$
    and the box space is the box subspace of the product box space induced
    by the underlying set.
\end{lemma}
\begin{lemma}
    Suppose that $f:(A,\mathcal{A})\to(B,\mathcal{B})$ is a
    box morphism. Then $Cl\circ f^{-1}=f^{-1}\circ Cl$.
    \begin{proof}
        \par Consider some subset $S$.
        If $\vec{x}\in Cl\circ f^{-1}[S]$,
        then for any box $\bobject(\vec{m},\vec{M})$, if
        $f^{-1}[S]\subseteq \bobject(\vec{m},\vec{M})$, then
        $\vec{x}\in \bobject(\vec{m},\vec{M})$.
        \par Suppose that $\vec{x}\notin f^{-1}\circ Cl(S)$.
        Then $f(\vec{x})\notin Cl(S)$.
        Meaning that there exists some $\bobject(\vec{m'},\vec{M'})$ 
        such that $S\subseteq \bobject(\vec{m'},\vec{M'})$, but
        $f(\vec{x})\notin \bobject(\vec{m'},\vec{M'})$.
        In particular, this implies that
        $\vec{x}\notin f^{-1}[\bobject(\vec{m'},\vec{M'})]$ and
        $f^{-1}[S]\subseteq f^{-1}[\bobject(\vec{m'},\vec{M'})]$.
        However, as $f$ is a box morphism, $f^{-1}[\bobject(\vec{m'},\vec{M'})]$
        is a box. This is a contradiction. Thus
        $Cl\circ f^{-1}[S]\subseteq f^{-1}\circ Cl$.
        \par Suppose that $\vec{x}\in f^{-1}\circ Cl(S)$.
        Then $f(\vec{x})\in Cl(S)$. Consider a box $\bobject(\vec{m},\vec{M})$
        such that $f^{-1}[S]\subseteq \bobject(\vec{m},\vec{M})$. If
        $x\notin \bobject(\vec{m},\vec{M})$, then 
        $g$
    \end{proof}
\end{lemma}
\begin{lemma}
    For any pullback diagram
    \[\begin{tikzcd}
        {(A\times B_{\pi',g},\mathcal{A}\times B_{A\times B_{\pi',g}})} && {(B,\mathcal{B})} \\
        \\
        {(A,\mathcal{A})} && {(C,\mathcal{C})}
        \arrow["\pi"{description}, from=1-1, to=1-3]
        \arrow["g"{description}, from=1-3, to=3-3]
        \arrow["f"{description}, from=1-1, to=3-1]
        \arrow["{\pi'}"{description}, from=3-1, to=3-3]
    \end{tikzcd}\]
    in $\mathbf{Box}$, the Beck-Chevalley condition holds.
    \begin{gather*}
        \exists_\pi\circ f^{-1}=g^{-1}\circ\exists_{\pi'}
    \end{gather*}
    \begin{proof}
        By lemma \ref{lem:box-pullback}, we have that $f,\pi$ are projections.
        Thus
        \begin{gather*}
            \exists_\pi\circ f^{-1}[\bobject(\vec{m},\vec{M})]=
            \exists_\pi(\{(\vec{a},\vec{b})\mid g(\vec{b})\in \pi'[\bobject(\vec{m},\vec{M})]\})=
            \\
            Cl(\pi[\{(\vec{a},\vec{b})\mid g(\vec{b})\in \pi'[\bobject(\vec{m},\vec{M})]\}])
            =
            Cl(\{\vec{b}\mid g(\vec{b})\in \pi'[\bobject(\vec{m},\vec{M})]\})=
            \\
            Cl(g^{-1}[\pi'[\bobject(\vec{m},\vec{M})]])
        \end{gather*}
        As $g$ is a box morphism,
        Let $\bobject(\vec{m},\vec{M})\in\mathcal{A}$.
        As $\exists_\pi$ is a left adjoint, to show that
        \begin{gather*}
            \exists_\pi\circ f^{-1}[\bobject(\vec{m},\vec{M})]
            \subseteq
            g^{-1}\circ\exists_{\pi'}[\bobject(\vec{m},\vec{M})]
        %     =
        % Cl(\pi\circ f^{-1}[\bobject(\vec{m},\vec{M})])=
        % Cl(\{\pi(\vec{x}):f(x)\in\bobject(\vec{m},\vec{M})\})
        \end{gather*}
        It suffices to show that
        \begin{gather*}
            f^{-1}[\bobject(\vec{m},\vec{M})]
            \subseteq
            \pi^{-1}\circ g^{-1}\circ\exists_{\pi'}[\bobject(\vec{m},\vec{M})]=(g\circ\pi)^{-1}\circ\exists_{\pi'}[\bobject(\vec{m},\vec{M})]
        \end{gather*}
        By the pullback property, it suffices to show that
        \begin{gather*}
            f^{-1}[\bobject(\vec{m},\vec{M})]
            \subseteq
            (\pi'\circ f)^{-1}\circ\exists_{\pi'}[\bobject(\vec{m},\vec{M})]=
            f^{-1}\circ\pi'^{-1}\circ\exists_{\pi'}[\bobject(\vec{m},\vec{M})]
        \end{gather*}
        As we showed that $\exists_{\pi'}$ is a left adjoint to $\pi'^{-1}$, 
        we have that $\pi'^{-1}\circ\exists_{\pi'}=Id$. Hence, we need to show that
        \begin{gather*}
            f^{-1}[\bobject(\vec{m},\vec{M})]
            \subseteq
            f^{-1}[\bobject(\vec{m},\vec{M})]
        \end{gather*}
        which is trivialy true.
        \par Next we need to show that
        \begin{gather*}
            g^{-1}\circ\exists_{\pi'}[\bobject(\vec{m},\vec{M})]
            \subseteq
            \exists_\pi\circ f^{-1}[\bobject(\vec{m},\vec{M})]            
        \end{gather*}
        Consider $\vec{x}\in g^{-1}\circ\exists_{\pi'}[\bobject(\vec{m},\vec{M})]$.
        Meaning that $g(\vec{x})\in\exists_{\pi'}[\bobject(\vec{m},\vec{M})]$.
        Hence, for every $\bobject(\vec{m'},\vec{M'})$
        such that $\pi'[\bobject(\vec{m},\vec{M})]\subseteq\bobject(\vec{m'},\vec{M'})$,
        we have that $g(\vec{x})\in \bobject(\vec{m'},\vec{M'})$.
        \par Suppose that $\vec{x}\notin\exists_\pi\circ f^{-1}[\bobject(\vec{m},\vec{M})]$.
        Then there exists some $\bobject(\vec{m'},\vec{M'})\in\mathcal{B}$
        such that $\pi\circ f^{-1}[\bobject(\vec{m},\vec{M})]\subseteq \bobject(\vec{m'},\vec{M'})$,
        but $\vec{x}\notin\bobject(\vec{m'},\vec{M'})$.
        By lemma \ref{lem:box-pullback}, we have that
        $f$ and $\pi$ are projections.
        Rewriting, we have that
        \begin{gather*}
            \pi\circ f^{-1}[\bobject(\vec{m},\vec{M})]=
            \{\vec{b}:g(\vec{b})=\pi'[\bobject(\vec{m},\vec{M})]\}
        \end{gather*}
    \end{proof}
\end{lemma}
\end{document}

% predicates and relations are only binary relations
% existential is exists b.(a,b)\in R and b\in C
% Add examples of types and of sets
% Beck-Chevalley condition is essentially that substitution commutes with the and operator

% boxes would not be graphs of affine transformations but rather boxes in higher dimension
% This is in the sense that they are predicates of higher dimensional